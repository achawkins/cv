\documentclass[10pt]{article}

\usepackage[margin=0.5in]{geometry}
\usepackage{enumitem}
\usepackage{titling}
\usepackage[T1]{fontenc}
\usepackage{color}
\usepackage{hyperref}

\setlength{\droptitle}{-0.75in}

\title{\textsc{\textbf{Andrew C. Hawkins}}\vspace{-13.7ex}}
\date{}
\author{}

\renewcommand{\labelitemii}{$\bullet$}
\pagenumbering{gobble}

\begin{document}

\maketitle

\section*{}
1032 Searay Ct.\hfill (561) 339-4065\\
Abingdon, MD 21009\hfill \texttt{\href{mailto:hawkinsandrewc@gmail.com}{hawkinsandrewc@gmail.com}}

\noindent\rule{\textwidth}{1pt}

\vspace{-2ex}

\section*{\textsc{Education}}
\textbf{Master of Science in Engineering, Applied Mathematics and Statistics}\hfill August 2015 -- May 2017
\begin{itemize}[noitemsep]
    \item[] \textit{Johns Hopkins University}, Baltimore, MD
\end{itemize}
\textbf{Bachelor of Science, Computational Mathematics}\hfill August 2010 -- December 2013
\begin{itemize}[noitemsep]
    \item[] \textit{Embry-Riddle Aeronautical University}, Daytona Beach, FL
\end{itemize}

\section*{\textsc{Professional Experience}}
\textbf{Senior Data Scientist}\hfill October 2019 -- present
\begin{itemize}
    \item[] \textit{The Chemours Company}, Wilmington, DE
    \item[] As a part of the digital team, I contribute to various projects ranging from yield optimization and document digitization to architecture design and DevOps.
\end{itemize}
\textbf{Data Scientist}\hfill June 2017 -- November 2019
\begin{itemize}
    \item[] \textit{Stanley Black \& Decker}, Towson, MD \footnote{I was responsible for every aspect of the software life cycle, from ideation and design to development and deployment, for all projects listed.}
    \item \textbf{Industry 4.0 Dashboard:} Modular, extensible web application for Industry 4.0 initiatives. Two modules currently exist. The first is for real-time logging, retrieval, and processing of environment, health, and safety data. The second provides an optimized production schedule to reduce overtime while keeping service levels high, provides predictive maintenance suggestions, and alerts unplanned downtime in real-time. The application manages thousands of users across global sites. Multiple languages are supported. The front end (SPA) and back end (REST) are decoupled.
    \item \textbf{Digital Gemba Board:} Central location for factories to store and display daily metrics and counter measures. Metrics can be defined arbitrarily and added to boards dynamically. Allows for better tracking of issues in the plant, resulting in more efficient production. Rolled out globally.
    \item \textbf{IoT Device Back End:} Web based back end to handle devices recording energy usage of various industrial machines. Built to handle thousands of devices with resolution as high as 100 samples per second. Included a rudimentary front end for extracting and visualizing the recorded data.
    \iffalse
    \item \textbf{Computer Vision Safety Assistant:} Assists operators when assembling progressive stamp die blocks. Uses a convolutional neural network to classify blocks as properly configured or not. Allows an operator to provide training examples to help annotate images. Resulted in a vast reduction in cost by informing the operator before continuing.
    \fi
    \item \textbf{Complexity Management Tracker:} Manages the SKU discontinuance pipeline. Allows users to submit SKUs to be discontinued and verifies all orphaned dependencies will be discontinued as well. Suggests similar SKUs based on similarity indexes of their bill of materials. Used manifold based clustering methods to build the similarities. Discovered product demand life cycles leading to automation of promotional periods and discontinuance.
\end{itemize}
\textbf{Data Scientist}\hfill January 2014 -- July 2015
\begin{itemize}
    \item[] \textit{Product Quest Manufacturing}, Daytona Beach, FL
    \item \textbf{Forecasting:} Forcast the demand for finished goods across all customers. Resulted in reduced inventory and higher service levels. Used various techniques from ARIMA to neural networks. Built software to automate the data ingestion and transformation steps to allow future forecasting.
\end{itemize}

\iffalse
\section*{\textsc{Teaching Experience}}
\textbf{Instructor}
\begin{itemize}[noitemsep]
    \item[] \textit{Johns Hopkins University}, Baltimore, MD
    \begin{itemize}[noitemsep]
        \item EN.550.112: Statistical Analysis II\hfill Summer 2016
    \end{itemize}
    \vspace{1ex}
    \item[] \textit{Daytona State College}, Daytona Beach, FL
    \begin{itemize}[noitemsep]
        \item MAT0028: Mathematics II\hfill Fall 2014
    \end{itemize}
\end{itemize}
\textbf{Teaching Assistant}
\begin{itemize}[noitemsep]
    \item[] \textit{Johns Hopkins University}, Baltimore, MD
    \begin{itemize}[noitemsep]
        \item[] Classes: Optimization in Finance, Discrete Mathematics, Mathematical Game Theory, Introduction to Optimization, and Mathematical Modeling and Consulting
    \end{itemize}
    \vspace{1ex}
    \item[] \textit{Embry-Riddle Aeronautical University}, Daytona Beach, FL
    \begin{itemize}[noitemsep]
        \item[] Classes: Introduction to Scientific Computing and Probability and Statistics
    \end{itemize}
\end{itemize}
\fi

\section*{\textsc{Languages and Technologies}}
\begin{itemize}
    \item[] \textbf{Advanced:} Python, Flask, Linux, SQL, MongoDB, JavaScript
    \item[] \textbf{Intermediate:} Vue.js, HTML, CSS, Octave (\textsc{Matlab}), \textsc{R}, \LaTeX{}, Fortran, Nginx
    \item[] \textbf{Interest/Learning:} Rust, Go, Haskell
\end{itemize}

\section*{\textsc{Publications}}
\begin{itemize}[leftmargin=*]
    \item[] [1] Smith, T. A. and \textbf{Hawkins, A.} (2015). An economic regression model to predict market movements. \textit{International Journal of Mathematics Trends and Technology}, \textit{28}(1), 1 -- 3. \texttt{\href{http://dx.doi.org/10.14445/22315373/IJMTT-V28P501}{doi:10.14445/22315373/IJMTT-V28P501}}
\end{itemize}

\end{document}
